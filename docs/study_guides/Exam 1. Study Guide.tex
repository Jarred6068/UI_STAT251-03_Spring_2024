\documentclass[18pt]{article}
\usepackage [left=25.4mm,top=25.4mm]{geometry}
\usepackage{amsmath}
\usepackage{amssymb}
\usepackage{graphicx}
%\usepackage{apacite}
\usepackage{url}
\usepackage{subfig}
\usepackage{csvsimple}
\usepackage{float}
\usepackage{lineno}
\usepackage[affil-it]{authblk}
\usepackage{setspace}
\usepackage{makecell} 
\usepackage{tikz}
\usepackage{csvsimple}
\usepackage{newfloat}
\usepackage{xcolor}
\usepackage{tabularx,booktabs}
\usepackage{multirow}
\usepackage{multicol}
\usepackage{array}














\begin{document}
	\title{Exam 1 Study Guide \\}
	\author{ STAT 251 Section 03 }
	\maketitle
	
	\begin{itemize}
		\item Know the “anatomy” of data (i.e., observations and variables, quantitative and categorical variables).
		
		\item Be able to identify the different types of quantitative and qualitative variables. 
		
		\item Understand samples versus populations.
		
		\item Understand statistics versus parameters.
		
		\item Understand descriptive versus inferential statistics.
		
		\item What is meant by a distribution?
		
		\item Be able to construct a frequency table from a small set of observations and know to find the frequency, relative frequency, and cumulative relative frequency 
		
		\item Know how to construct a dot plot, stem plot, and a histogram given a small set of observations.
		
		\item Be able to compute a mean, median, and mode given a small set of observations. Note that you should also know how to compute a mean using a frequency table.
		
		\item Know how to compute a proportion
		
		\item Know how to find the modal category of a qualitative variable
		
		\item Be able to compute and interpret the variance and standard deviation given a small set of observations
		
		\item Know how to compute the five-number summary of a variable.
		
		\item Understand how plot a cumulative distribution and use it to find the percentiles or quartiles of a distribution. 
		
		\item Know how a box plot is constructed from a five number summary.
		
		\item Know how to interpret the shape (symmetry, skew, modality) of a distribution and how it is related to the mean and median
		
		\item Know how to use the $1.5\times IQR$ to identify outliers.
		
		\item Understand what it means to say that a summary measure is resistant to outliers, and which summary measures we have discussed that are resistant and which are not
		
		\item Be sure you understand the notation (i.e., symbols) we have used so far (e.g., $n,N,s,s^2,\bar{x}, \mu,\sigma,\sigma^2$
		
	\end{itemize}
	
	The following formulas will be provided on the the exam
	\[\bar{x} = \frac{1}{n} \sum_{i=1}^n x_i, \ \ \bar{x} = \frac{1}{n}\sum_{x} xF(x), \ \  \bar{x} = \sum_{x} xRF(X)\]
	
	\[s^2 = \frac{1}{n-1}\sum_{i=1}^n (x_i - \bar{x})^2, \ \ s = \sqrt{\frac{1}{n-1}\sum_{i=1}^n (x_i - \bar{x})^2}\]
		
	\[ \text{range}(x) = \text{min}(x) - \text{max}(x), \ \ IQR = Q3 - Q1 \]
	
	\[x < Q1 - 1.5 \times (Q3 - Q1), \ \  x > Q3 - 1.5\times(Q3-Q1)\]
\end{document}

